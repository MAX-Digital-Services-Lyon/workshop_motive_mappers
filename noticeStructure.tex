% Configuration des marges
\usepackage[top=0.8in, bottom=0.8in, left=0.8in, right=0.8in]{geometry}

% Personnalisation des titres de section et table des matières
\usepackage{sectsty}
\usepackage{tocloft}
\usepackage[pdftex, pdfauthor={Kevin Delfour}, pdftitle={En Quête d'Expérience : Le chemin du.de la jeune Développeur·euse}, pdfkeywords={développeur, développeuse, expérience, quête, open source}, hidelinks]{hyperref}
\usepackage{chngcntr}
\counterwithout{chapter}{part}

% Cadres, tailles de police et en-têtes/pieds de page personnalisés
\usepackage{framed}
\usepackage{anyfontsize}
\usepackage{fancyhdr}

% Police Garamond et paramètres de langue
\usepackage{ebgaramond}
\usepackage[french]{babel}
\usepackage[T1]{fontenc}
\usepackage{float}

% Couleurs personnalisées et graphiques TikZ
\usepackage[dvipsnames]{xcolor}
\usepackage{tikz}
\usepackage{microtype}
\usepackage{GS1}
\usepackage{qrcode}
\usetikzlibrary{calc}

\usepackage[labelformat=empty]{caption}
\captionsetup{justification=raggedleft, font={tiny, color=gray}, singlelinecheck=false}

% Contenu : texte aléatoire, boucles, images et commandes supplémentaires
\usepackage{lipsum}
\usepackage{pgffor}
\usepackage{graphicx}
\usepackage{etoolbox}

% Configuration de la mise en page
\renewcommand{\familydefault}{\sfdefault}
\setlength{\parskip}{\baselineskip}
\setlength{\parindent}{0pt}

% Configuration de la table des matières
\setcounter{tocdepth}{2}
\renewcommand{\cftpartfont}{\normalfont\bfseries\Large}
\renewcommand{\cftchapfont}{\normalfont\bfseries}
\renewcommand{\cftsecfont}{\normalfont}
\renewcommand{\cftsubsecfont}{\normalfont}
\renewcommand{\cftsubsubsecfont}{\normalfont\small}

% Configuration des en-têtes et pieds de page personnalisés
\pagestyle{fancy}
\fancyhead{}
\fancyhead[LO, L]{ \color{gray}\scriptsize\leftmark}
\fancyfoot{}
\fancyfoot[R]{\scriptsize\thepage}
\renewcommand{\headrulewidth}{0pt}
